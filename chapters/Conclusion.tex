\chapter{CONCLUSION}\label{chap:conclusion}
In this dissertation, applications of Reverse Non-Equilibrium Molecular Dynamics and the Langenvin Hull method were applied to systems containing gold nanoparticles. Gold nanoparticles have well characterized optical properties, but less-well explored thermal properties, which are typically important in the period after electronic excitation. The simulations in previous chapters explored the effect of ligand rigidity, gold particle morphology, and staple motifs in nanoarrays using molecular dynamics.

Understanding how the capping agent, or ligand, on gold nanoparticles interacts with the particle and solvent environment is essential to predicting behavior of nanoparticles in a given system. Adding rigidity to the ligand layer changed the penetration accessibility of the interfacial solvent. With a more rigid ligand, the solvent generally has an increased penetration into the ligand layer. This leads to a stronger vibrational overlap of the ligand and solvent, and thus more rapid thermal transport. Additionally, the addition of the ligand layer aids interfacial thermal conductivity. The sulfur on the thiolate ligand acts as a conduit for heat from the particle to the solvent.

Though the ligand layer is experimentally interesting and has promising characteristics for tailoring a nanoparticle for desired thermal and optical properties, a bare particle gives insight to how the morphology of the gold particle affects thermal transport.
From simulations of spheres, cuboctahedra, and icosahedra; the surface structure of the gold particle is essential to thermal transport from bare particles.
Furthermore, work in chapter 3 displayed that the underlying lattice carries important information on the low frequency phonon modes. 

In an effort examine both the ligand and undercoordination of gold atoms at the surface within the same system, chapter 4 investigates the effect of the staple motif on \ce{Au144PET60} nanoclusters.
The nanoparticles were examined in two different solvents; one polar and one non-polar; in single particle systems and in nanoarrays.
While the array systems have so far yielded inconclusive results, the single particle systems display interesting behavior with respect to the ligand/solvent interactions. 
While the staple motif might increase the conductivity through the undercoordinated gold atoms in the ligand, the solvent is ultimately the limiting factor in thermal transport. 
Additionally, the solvent thermal conductivity due to the partial volume becomes the dominating factor in the predictive equation for thermal conductivity through an array of \ce{Au144PET60} particles.

In the future, biologically relevant systems would be an interesting avenue to explore. 
With the new polarizable metal model (DR-EAM), being developed in the group, the gold substrate would be able to more appropriately interact with polar molecules.
Thus, a good starting system would be a gold slab displaying (111) facets in water.

To systematically determine how polarization affects thermal conductivity in a simple gold/water system, each of the components should be tested individually. Therefore the following systems should be examined: \ce{H2O}/gold (with and without polarization) and fluctuating-charge \ce{H2O}/gold (with and without fluctuating charge).
If the polarization of the gold and water effects the the interfacial thermal conductivity of the system, moving forward with biologically relevant systems would be the next course of action.
Various interesting ligands such as: thiolated PEG, citrate, Cetyl trimethylammonium bromide(CTAB), and different carboxylic acid moieties; could be tested on planar and spherical systems. The differences in these could then be identified to tailor nanoparticles for different thermal needs.

In addition to the further work needed in the gold nanoparticle systems, the trend in pure solvents would be an interesting avenue to explore.
In chapter 4, the two solvents (dichloromethane and toluene) had very different thermal conductivity values at the three box lengths simulated but had the same infinite box length bulk thermal conductivity.
Dicholormethane, with a density of $\approx$ 2x that of toluene, had a clear box length dependence, while toluene did not. 
It would be interesting to follow these simulations with a range of different common solvents, with known thermal conductivities, and look not only at the box dependences but a normal mode analysis.
In the more dense fluids it is possible that the imaginary frequencies would be more heavily populated, leading to higher bulk thermal conductivities.

The work presented in this work has shown significant progress in identifying the trends that are important for controlling the thermal properties of gold nanoparticles. Moving forward, work with polarizable force fields, biologically relevant ligands and capping agents, and elucidating trends in solvent thermal conductivity would move nanoparticle design forward.
