\documentclass[aps,jcp,preprint,showpacs,superscriptaddress,groupedaddress]{revtex4-1}
\usepackage{graphicx}  % needed for figures
\usepackage{dcolumn}   % needed for some tables
\usepackage{bm}        % for math
\usepackage{amssymb}   % for math
\usepackage{booktabs}
\usepackage[english]{babel}
\usepackage{multirow}
\usepackage{times}
\usepackage[version=3]{mhchem}
\usepackage{lineno}
\usepackage{gensymb}
\usepackage{multirow}

\begin{document}

\title{Supporting Information for: Interfacial Thermal Conductance of Thiolate-Protected
  Gold Nanospheres}
\author{Kelsey M. Stocker}
\author{Suzanne M. Neidhart}
\author{J. Daniel Gezelter}
\email{gezelter@nd.edu}
\affiliation{Department of Chemistry and Biochemistry, University of
  Notre Dame, Notre Dame, IN 46556}
\date{\today}

\begin{abstract}
  This document supplies force field parameters for the united-atom
  sites, bond, bend, and torsion parameters, as well as the cross
  interactions between the united-atom sites and the gold atoms. These
  parameters were used in the simulations presented in the main text.
\end{abstract}


\maketitle

Gold -- gold interactions were described by the quantum Sutton-Chen
(QSC) model.\cite{Qi:1999ph} The hexane solvent is described by the
TraPPE united atom model,\cite{TraPPE-UA.alkanes} where sites are
located at the carbon centers for alkyl groups. Bonding interactions
were used for intra-molecular sites closer than 3 bonds. Effective
Lennard-Jones potentials were used for non-bonded interactions.

\begin{table}[h]
\bibpunct{}{}{,}{n}{}{,}
\centering
\caption{Non-bonded interaction parameters (including cross interactions with Au atoms). \label{tab:atypes}}
\begin{tabular}{ c|cccccl }
 \toprule
Site & mass & $\sigma_{ii}$ & $\epsilon_{ii}$ & $\sigma_{\ce{Au}-i}$ & $\epsilon_{\ce{Au}-i}$  & source \\
     & (amu)& (\AA)        & (kcal/mol)     & (\AA)             &  (kcal/mol)          &  \\
 \colrule
 \ce{CH3}    & 15.04    & 3.75  & 0.1947 & 3.54   & 0.2146 & Refs. \protect\cite{TraPPE-UA.alkanes}, \protect\cite{vlugt:cpc2007154} and \protect\cite{landman:1998}\\
 \ce{CH2}    & 14.03    & 3.95  & 0.09141& 3.54   & 0.1749 & Refs. \protect\cite{TraPPE-UA.alkanes}, \protect\cite{vlugt:cpc2007154} and \protect\cite{landman:1998}\\
 CHene       & 13.02    & 3.73  & 0.09340& 3.4625 & 0.1680 & Refs. \protect\cite{TraPPE-UA.alkylbenzenes}, \protect\cite{vlugt:cpc2007154} and \protect\cite{landman:1998}\\
 S           & 32.0655  & 4.45  & 0.2504 & 2.40   & 8.465  & Refs. \protect\cite{landman:1998} ($\sigma$) and \protect\cite{vlugt:cpc2007154} ($\epsilon$) \\
 CHar        & 13.02    & 3.695 & 0.1004 & 3.4625 & 0.1680 & Refs. \protect\cite{TraPPE-UA.alkylbenzenes} and \protect\cite{vlugt:cpc2007154}\\
 \ce{CH2ar}  & 14.03    & 3.695 & 0.1004 & 3.4625 & 0.1680 & Refs. \protect\cite{TraPPE-UA.alkylbenzenes} and \protect\cite{vlugt:cpc2007154}\\
 \botrule
\end{tabular}
\bibpunct{[}{]}{,}{n}{,}{,}
\end{table}

The TraPPE-UA force field includes parameters for thiol
molecules\cite{TraPPE-UA.thiols} which were used for the
alkanethiolate molecules in our simulations.  To derive suitable
parameters for butanethiolate adsorbed on Au(111) surfaces, we adopted
the S parameters from Luedtke and Landman\cite{landman:1998} and
modified the parameters for the CTS atom to maintain charge neutrality
in the molecule.

Bonds are typically rigid in TraPPE-UA, so although we used
equilibrium bond distances from TraPPE-UA, for flexible bonds, we
adapted bond stretching spring constants from the OPLS-AA force
field.\cite{Jorgensen:1996sf}

\begin{table}[h]
\bibpunct{}{}{,}{n}{}{,}
\centering
\caption{Bond parameters. \label{tab:bond}}
\begin{tabular}{ cc|ccl }
 \toprule
 $i$&$j$ & $r_0$ & $k_\mathrm{bond}$ & source \\
    &    & (\AA) & $(\mathrm{~kcal/mole/\AA}^2)$ & \\
 \colrule
\ce{CH3}   & \ce{CH3} &	1.540	& 536  & Refs. \protect\cite{TraPPE-UA.alkanes} and \protect\cite{Jorgensen:1996sf}\\
\ce{CH3}   & \ce{CH2} &	1.540	& 536  & Refs. \protect\cite{TraPPE-UA.alkanes} and \protect\cite{Jorgensen:1996sf} \\
\ce{CH2}   & \ce{CH2} &	1.540	& 536  & Refs. \protect\cite{TraPPE-UA.alkanes} and \protect\cite{Jorgensen:1996sf} \\
CHene      & CHene    & 1.330   & 1098 & Refs. \protect\cite{TraPPE-UA.alkylbenzenes} and \protect\cite{Jorgensen:1996sf}\\
\ce{CH3}   & CHene    & 1.540   & 634  & Refs. \protect\cite{TraPPE-UA.alkylbenzenes} and \protect\cite{Jorgensen:1996sf} \\
\ce{CH2}   & CHene    & 1.540   & 634  & Refs. \protect\cite{TraPPE-UA.alkylbenzenes} and \protect\cite{Jorgensen:1996sf} \\
S          & \ce{CH2} & 1.820   & 444  & Refs. \protect\cite{TraPPE-UA.thiols} and \protect\cite{Jorgensen:1996sf} \\
CHar       & CHar     & 1.40    & 938  & Refs. \protect\cite{TraPPE-UA.alkylbenzenes} and \protect\cite{Jorgensen:1996sf} \\
CHar       & \ce{CH2} & 1.540   & 536  & Refs. \protect\cite{TraPPE-UA.alkylbenzenes} and \protect\cite{Jorgensen:1996sf}\\
CHar       & \ce{CH3} & 1.540   & 536  & Refs. \protect\cite{TraPPE-UA.alkylbenzenes} and \protect\cite{Jorgensen:1996sf}\\
\ce{CH2ar} & CHar     & 1.40    & 938  & Refs. \protect\cite{TraPPE-UA.alkylbenzenes} and \protect\cite{Jorgensen:1996sf} \\
S          & CHar     &	1.80384	& 527.951 & This Work \\
 \botrule
\end{tabular}
\bibpunct{[}{]}{,}{n}{,}{,}
\end{table}

To describe the interactions between metal (Au) and non-metal atoms,
potential energy terms were adapted from an adsorption study of alkyl
thiols on gold surfaces by Vlugt, \textit{et
  al.}\cite{vlugt:cpc2007154} They fit an effective pair-wise
Lennard-Jones form of potential parameters for the interaction between
Au and pseudo-atoms CH$_x$ and S based on a well-established and
widely-used effective potential of Hautman and Klein for the Au(111)
surface.\cite{hautman:4994}

Parameters not found in the TraPPE-UA force field for the
intramolecular interactions of the conjugated and the penultimate
alkenethiolate ligands were calculated using constrained geometry
scans using the B3LYP functional~\cite{Becke:1993kq,Lee:1988qf} and
the 6-31G(d,p) basis set. Structures were scanned starting at the
minimum energy gas phase structure for small ($C_4$) ligands.  Only
one degree of freedom was constrained for any given scan -- all other
atoms were allowed to minimize subject to that constraint.  The
resulting potential energy surfaces were fit to a harmonic potential
for the bond stretching,
\begin{equation}
V_\mathrm{bond} = \frac{k_\mathrm{bond}}{2} \left( r - r_0 \right)^2,
\end{equation}
and angle bending potentials,
\begin{equation}
V_\mathrm{bend} = \frac{k_\mathrm{bend}}{2} \left(\theta - \theta_0\right)^2.
\end{equation}
Torsional potentials were fit to the TraPPE torsional function,
\begin{equation}
V_\mathrm{tor} = c_0 + c_1  \left(1 + \cos\phi \right) + c_2  \left(1 - \cos 2\phi \right) + c_3  \left(1 + \cos 3 \phi \right).
\end{equation}

For the penultimate thiolate ligands, the model molecule used was
2-Butene-1-thiol, for which one bend angle (\ce{S-CH2-CHene}) was
scanned to fit an equilibrium angle and force constant, as well as one
torsion (\ce{S-CH2-CHene-CHene}).  The parameters for these two
potentials also served as model for the longer conjugated thiolate
ligands which require bend angle parameters for (\ce{S-CH2-CHar}) and
torsion parameters for (\ce{S-CH2-CHar-CHar}).

For the $C_4$ conjugated thiolate ligands, the model molecule for the
quantum mechanical calculations was 1,3-Butadiene-1-thiol.  This
ligand required fitting one bond (\ce{S-CHar}), and one bend angle
(\ce{S-CHar-CHar}).

The geometries of the model molecules were optimized prior to
performing the constrained angle scans, and the fit values for the
bond, bend, and torsional parameters were in relatively good agreement
with similar parameters already present in TraPPE.


\begin{table}[h]
\bibpunct{}{}{,}{n}{,}{,}
\centering
\caption{Bend angle parameters. The central atom in the bend is atom $j$.\label{tab:bend}}
\begin{tabular}{ ccc|ccl }
\toprule
 $i$&$j$&$k$ & $\theta_0$ & $k_\mathrm{bend}$ & source\\
    &   &    & ($\degree$) & (kcal/mol/rad\textsuperscript{2}) & \\
 \colrule
\ce{CH2} & \ce{CH2} & S         & 114.0   &   124.20& Ref. \protect\cite{TraPPE-UA.thiols}\\ 
\ce{CH3} & \ce{CH2} & \ce{CH2}  & 114.0   &   124.20& Ref. \protect\cite{TraPPE-UA.thiols}\\ 
\ce{CH2} & \ce{CH2} & \ce{CH2}  & 114.0   &   124.20& Ref. \protect\cite{TraPPE-UA.thiols}\\ 
CHene    & CHene    & \ce{CH3}  & 119.7   &   139.94& Ref. \protect\cite{TraPPE-UA.alkylbenzenes}\\ 
CHene    & CHene    & \ce{CH2}  & 119.7   &   139.94& Ref. \protect\cite{TraPPE-UA.alkylbenzenes}\\ 
\ce{CH2} & \ce{CH2} & CHene     & 114.0   &   124.20& Ref. \protect\cite{TraPPE-UA.alkylbenzenes}\\ 
CHar     & CHar     & CHar      & 120.0   &   126.0 & Refs. \protect\cite{TraPPE-UA.alkylbenzenes} and \protect\cite{Jorgensen:1996sf}\\ 
CHar     & CHar     & \ce{CH2}  & 120.0   &   140.0 & Refs. \protect\cite{TraPPE-UA.alkylbenzenes} and \protect\cite{Jorgensen:1996sf}\\
CHar     & CHar     & \ce{CH3}  & 120.0   &   140.0 & Refs. \protect\cite{TraPPE-UA.alkylbenzenes} and \protect\cite{Jorgensen:1996sf}\\
CHar     & CHar     & \ce{CH2ar}& 120.0   &   126.0 & Refs. \protect\cite{TraPPE-UA.alkylbenzenes} and \protect\cite{Jorgensen:1996sf}\\
S        & \ce{CH2} & CHene     & 109.97  &  127.37 & This work  \\
S        & \ce{CH2} & CHar      & 109.97  &  127.37 & This work  \\
S        & CHar     & CHar      & 123.911 & 138.093 & This work  \\
 \botrule
\end{tabular}
\bibpunct{[}{]}{,}{n}{,}{,}
\end{table}

\begin{table}[h]
\bibpunct{}{}{,}{n}{,}{,}
\centering
\caption{Torsion parameters. The central atoms for each torsion are atoms $j$ and $k$,
  and wildcard atom types are denoted by ``X''.  All $c_n$ parameters
  have units of kcal/mol. The torsions around carbon-carbon double bonds
  are harmonic and assume a trans (180$\degree$) geometry.  The force
  constant for this torsion is given in $\mathrm{kcal~mol~}^{-1}\mathrm{degrees}^{-2}$.  \label{tab:torsion}}
\begin{tabular}{ cccc|ccccl }
\toprule
 $i$&$j$&$k$&$l$& $c_0$&$c_1$& $c_2$ & $c_3$ & source\\
 \colrule
\ce{CH3} & \ce{CH2} & \ce{CH2} & \ce{CH2} & 0.0     & 0.7055   & -0.13551 &  1.5725    & Ref. \protect\cite{TraPPE-UA.alkanes}\\
\ce{CH2} & \ce{CH2} & \ce{CH2} & \ce{CH2} & 0.0     & 0.7055   & -0.13551 &  1.5725    & Ref. \protect\cite{TraPPE-UA.alkanes}\\
\ce{CH2} & \ce{CH2} & \ce{CH2} & S        & 0.0     & 0.7055   & -0.13551 &  1.5725    & Ref. \protect\cite{TraPPE-UA.thiols}\\ \colrule
X        & CHene    & CHene    & X        & \multicolumn{4}{c}{\multirow{2}{*}{$V = \frac{0.008112}{2} (\phi - 180.0)^2$}} & \multirow{2}{*}{Ref. \protect\cite{TraPPE-UA.alkylbenzenes}} \\
X        & CHar     & CHar     & X        &         & & & & \\ \colrule
\ce{CH2} & \ce{CH2} & CHene    & CHene    & 1.368   & 0.1716   & -0.2181  &  -0.56081  & Ref. \protect\cite{TraPPE-UA.alkylbenzenes}\\
\ce{CH2} & \ce{CH2} & \ce{CH2} & CHene    & 0.0     & 0.7055   & -0.13551 &   1.5725   & Ref. \protect\cite{TraPPE-UA.alkylbenzenes}\\
CHene    & CHene    & \ce{CH2} & S        & 3.20753 & 0.207417 & -0.912929&  -0.958538 & This work \\
CHar     & CHar     & \ce{CH2} & S        & 3.20753 & 0.207417 & -0.912929&  -0.958538 & This work \\
 \botrule
\end{tabular}
\bibpunct{[}{]}{,}{n}{,}{,}
\end{table}

\newpage
\bibliographystyle{aip}
\bibliography{NPthiols}
\end{document}
